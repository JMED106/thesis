\chapter{\bf \sffamily Generalization of the QIF-FR model}
\label{cha:appA}

% \section{Firing rate model for conductance-based spiking neurons}

% % Following the general expression  for the QIF neuron
% % \eqref{eq:qif_general}, we can now define the conductance-based QIF
% % neuron as:
% \begin{equation}
% \label{eq:1}
% \tau \dot{v}_j = v_j^2 + \eta_j - g \tau S \left( t \right) \left(
%   v - v_{E,j} \right),
% \end{equation}
% where $g$ corresponds to the conductance of the synapses and
% $v_{E,j}$ denotes the reversal potential of the neuron $j$, which may
% be distributed in a similar fashion as the external currents $\eta_j$.

% {\color{red} We explicitly do the derivation ... }
% By applying the same procedure as before we obtain the QIF-firing-rate
% equations for the population of neurons:
% \begin{subequations}
%   \label{eq:conductance_fr}
% \begin{eqnarray}
% \label{eq:conductance_fr-R}
% \tau \dot R & = & \frac{\Delta}{\tau \pi} + 2 R V - g \tau R \left( R
%                   - \frac{\Gamma }{\tau\pi}\right) \\
%   \label{eq:conductance_fr-V}
% \tau \dot V & = & V^2 + \bar{\eta} - \left( \pi \tau R \right)^2 - g
%                   \tau R \left( V - V_E \right).
% \end{eqnarray}
% \end{subequations}

% \paragraph{Rescaled equations:} rescaling $\Delta$ and $\tau$ in the
% following way:
% \begin{eqnarray}
% \label{eq:3}
% R = \frac{\sqrt{\Delta}}{\tau} r; \quad V = \sqrt{\Delta} v; \quad t =
% \frac{\tau}{\sqrt{\Delta}}\tilde{t}. \\
% \bar{\eta} = \Delta \tilde{\eta}; \quad \Gamma = \Delta \gamma; \quad
% V_E = \sqrt{\Delta} v_E, \nonumber
% \end{eqnarray}
% gives
% \begin{subequations}
% \label{eq:conductance_fr-rescaled}
% \begin{eqnarray}
% \label{eq:conductance_fr-R-r}
% \dot r & = & \frac{1}{\pi} + 2 rv - g r \left( r
%                   - \frac{\gamma}{\pi}\right) \\
%   \label{eq:conductance_fr-V-r}
% \dot v & = & v^2 + \tilde{\eta} - \left( \pi  r \right)^2 - g
%                   r \left( v - v_E \right).
% \end{eqnarray}
% \end{subequations}

% \paragraph{Without external current heterogeneity:}  setting $\Delta =
% 0$ makes easy to compute the fixed points. We may re-scale the equations
% again:
% \begin{eqnarray}
% \label{eq:3}
% R = \frac{\Gamma}{\tau} r; \quad V =  \Gamma v; \quad t =
% \frac{\tau}{\Gamma}\tilde{t}. \\
% \bar{\eta} = \Gamma^2 \tilde{\eta}; \quad
% V_E = \Gamma v_E, \nonumber
% \end{eqnarray}
% which leads to:
% \begin{subequations}
% \label{eq:conductance_fr-rescaled2}
% \begin{eqnarray}
% \label{eq:conductance_fr-R-r-2}
% \dot r & = & 2 rv - g r \left( r
%                   - \frac{1}{\pi}\right) \\
%   \label{eq:conductance_fr-V-r-2}
% \dot v & = & v^2 + \tilde{\eta} - \left( \pi  r \right)^2 - g
%                   r \left( v - v_E \right).
% \end{eqnarray}
% \end{subequations}




%  Eq. \eqref{eq:conductance_fr-R} gives two possible fixed
% points:
% \begin{subequations}
%   \label{eq:qif-cond-fixed}
%   \begin{align}
%     \label{eq:qif-cond-fixed1}
%     \left(r_{*}, v_{*} \right) &=\left( 0, - \sqrt{-\tilde \eta} \right).\\
%     \label{eq:qif-cond-fixed2}
%     \left(r_{*}, v_{*} \right) &=\left(\frac{v_Eg \pm \sqrt{\left( v_Eg \right)^2 + \left( g^2 + \pi^2
%         \right)\left( 4\eta + g^2/\pi^2 \right)}}{2 +
%       \frac{g^2}{2\pi^2}},  \frac{g}{2\pi} \left( \pi r - 1 \right) \right).
%   \end{align}
% \end{subequations}

% \paragraph{Without reversal potential heterogeneity:} hola.


%%% Local Variables: 
%%% mode: latex
%%% ispell-local-dictionary: "american"
%%% TeX-master: "main"
%%% End: 
