\chapter*{\bf {Introduction}}
\label{cha:intro}
\addcontentsline{toc}{chapter}{\bf \uppercase{Introduction}}

% Historical aperture: we must focus on mathematical neuroscience but
% also mention the theories that attribute a computational role to
% oscillatory dynamics in the brain.

% Don't forget to mention the advanced in dynamical systems and
% statistical physics which allowed so many quantitative studies in
% theoretical neuroscince

% We must do a brief review of the population models (firing rate models: WC
% type models; density models: FP, ...;)

Since the pioneering discoveries of Santiago Ramon y Cajal and Camilo Golgi in
the late 19th century which established the foundations of what we nowadays know as
Neuroscience, extensive work has been done in order to understand the
laws that shape neural dynamics. As in almost any natural science,
the aim of scientists toward universal laws explaining our
observations have push, from the very beginning, towards mathematical
descriptions of the observed phenomena. Early in the 20th century,
Louis Lapicque introduced what is considered as the precursor  of
the famous integrate-and-fire model neuron \citep{Brunel2007}
\citep[see also][]{Brunel2007a,Abbott1999}, giving an intuitive view of the polarization
process of neurons and the generation of the  
first spike after stimulus onset ---what we know as
the action potential. Theoretical
neuroscience was born and following his work many phenomenological
models describing the onset of the action potential were developed
\citep{Hill1936,McCulloch1943,Stein1965,Geisler1966,Weiss1966,Stein1967b}.

In the 1950's Hodgkin and Huxley published the first detailed biophysical model
of the action potential \citep{Hodgkin1952}, a system of three differential equations
which modeled the electrical currents across the cell membrane leading to action potentials in
the squid's giant axon, and for which they received the Nobel price in
1963. Their almost 30 year investigation not only proved that 
detailed biophysical ---yet simple and elegant--- models of neural
dynamics were possible,  but
also drove them to the hypothesis of the existence of ionic channels,
which were confirmed only few decades later. 

Further development in the biophysical principles of neural dynamics
was made with the addition of more cellular elements to the models, such as
different classes of ionic channels and pumps, synapses,
etc. Currently, detailed biophysical modeling of single neurons
remains  a very active research area where sophisticated
models are continuously expanding our knowledge about neural
mechanisms. On the other hand, the necessity of decreasing the complexity
of neural systems lead many theoreticians towards simplified models
of the neuron, where only the underlying mechanisms that generate the
action potential were studied. We refer to these phenomenological
models as point neurons or \textbf{spiking neuron} models. The first
of its kind was probably that developed by Lapicque that
later evolved into the \textit{leaky integrate-and-fire} (LIF) neuron model
\citep{Stein1965,Kni72}, which can be mathematically 
derived as a limit case of the Hodgkin-Huxley model \citep[see~for~example][]{Gerstner2014}. Since the
outbreak of computer driven simulations, their elegance and
simplicity has make them a widely used tool in the study of
principles of neural information processing.

Yet, the complexity of each neuron added to their outstanding
numerosity,
makes the brain one of the most complex systems ever
studied. A universe inside our heads
is still waiting to be discovered.
Even if we would use one of the simplest phenomenological
model neurons, and without taking into account any
complex network structure, a simple model of the brain would consist on a system of approximately
$86\times 10^9$ differential equations --each describing the dynamics
of a single neuron. A system
impossible to work with, nor in the  20th century nor nowadays; even with
the incredible computational power at our hands. With this paradigm in
mind, models of population average activity started to appear, first
in the 1950's by the hands of \citet{Beu56}, and later in the
1960's by 
\citet{Griffith1963}. They developed the first continuum
approximations of neural activity by making some statistical
assumptions, following a methodology closely related to that used in
statistical mechanics in connection with
thermodynamics. The aim was to 
create low dimensional models capable of capturing the essential
collective properties of neural populations, yet simple enough to provide a
mathematically tractable framework for their study. 

In parallel, the rapid progress made in experimental neuroscience
with studies such as those carried out by Mountcastle and Hubel and
Wiesel in the somato-sensory and visual cortices of cats and monkeys
\citep{Mountcastle1957,Hubel1962,Hubel1968}, gave support to the
original hypothesis of Beurle. During the 1970's all
these accumulation of experimental evidence 
inspired the work of many researchers, including
\citet{Wilson1972,Wilson1973}, \citet{Kni72} and
\citet{Amari1972,Ama74,Amari1977} among others, towards the
development of  population models, also known as \textit{neural mass}
models. ...
\textbf{Firing Rate} models for the activity of neural population which
became the canonical models for the study of large populations of
neurons. These models consisted in systems of differential (or
integro-differential) equations which could be $i)$ mathematically
studied using the tools provided by dynamical systems theory and $ii)$
easily solved using numerical methods in computer-driven simulations.

\textit{(intro of the firing rate model)The model was thought from the
  pragmatic point of view that sensory 
information is introduced into the nervous system in the form of large
spatio-temporal activity which involved too many neurons to be studied
from the single neuron level. In addition, they exploited the
hypothesis that neurons alone were unreliable units of information
processing, and thus they assumed that any reliable observable should
involve large population of neurons.
}
\textit{Brief description of what they are and their
  mathematical representation (differential or integro-differential equations)}

During the last 50 years, their equations were adapted or expanded to
model different network configurations, to include synaptic kinetics,
transmission delays, etc. They are used not only to model population
of neurons but also average activity of single neurons (measured as
trial-to-trial averages). Furthermore, they can be either used to
describe discrete populations or spatially distributed systems, such
as orientation selectivity or head direction systems. The latter case
is modeled by means of a partial integro-differential version of the
same equations.

However, in general the connection between the macroscopic models:
such as firing rate models, and their microscopic counterpart:
populations of spiking neurons, has been so far done in an heuristic
manner. In other words, the derivation of such equations does not
follow a proper reduction of the microscopic neural system. Moreover,
traditional firing rate models assume microstates  where neurons'
activity is completely uncorrelated, and therefore they usually
fail to describe any dynamical phenomena occurring as a consequence
of synchronous firing, or even highly correlated activity. These
include fast transients and some types of persistent oscillations
observed in their homologous spiking neuron networks. In the following
sections we briefly review the WC firing rate equations and describe
the quadratic-integrate-and-fire (QIF) model neuron which we will use
in the rest of the work.
\newline 

In Chapter I we apply these models to simulate excitatory and inhibitory
neural networks and show some of the discrepancies that exist between
both approaches. In addition, we present a novel firing rate model,
exactly derived from the quadratic-integrate-and-fire model neuron
which will solve such discrepancies and allow us to perform further
dynamical studies. In Chapter II and III we show how this new FR model
give precise results of networks of spiking neurons and we extract
some interesting mechanistic conclusions which we discuss in the final
section of this work.

\cleardoublepage

%%% Local Variables: 
%%% mode: latex
%%% ispell-local-dictionary: "american"
%%% TeX-master: "main"
%%% End: 
