\chapter{\bf From single neuron dynamics to population dynamics}
\label{cha:chap1}

\epigraph{``Quote''}{--- \textup{Author}, Title of Source.}

\section{Introduction}

\begin{itemize}
\item Seminal works of Wilson and Cowan in 1972-73, Amari, Nunez
\item Even a small piece of cortex has millions of neurons: extremely
complex and large networks.
\item Neurons themselves are extremely complex biological systems.
\item Aim: to be able to study the dynamics of networks of neurons and
understand the underlying mechanisms.
However, the dynamics of even a single neuron are highly non-linear
and in general very irregular due to the interaction with the rest
of the network and the extracellular environment.
 in the last 50 years a variety of mathematical models have been
 developed To be able to catch the most relevant features of the
 dynamics of neurons, without compromising
the viability of the study of  larger systems.  
Even though, and due to the extremely high amount of neurons present
in any relevant cortical circuit, the study of population dynamics
generally require the use of mean-field models which average the
dynamics of large number of neurons and provide a macroscopic
description by means of measurable magnitudes.
\item LFP, FMRI, EEG, FR
\item \textbf{New macroscopic model, able to capture the dynamics of
    populations of correlated neurons: synchronization.}
\end{itemize}

\subsection{Spiking neuron models}

\begin{itemize}
\item Hodgkin-Huxley, and IF models.
\item Conductance-based and current-based IF models.
\item Type I and type II. Transfer function.
\item Comparison of different spiking neuron models: their viability
  and biological plausibility.
\end{itemize}

\subsection{Population dynamics: Firing rate and neural field models}

\begin{itemize}
\item Macroscopic magnitudes involving the activity of neurons.
\item Firing rate models: Wilson-Cowan type FR models. Heuristically
  obtained. Asynchronous activity of neurons as a fundamental
  assumption. They are not able to capture the transient dynamics of
  neural populations. (Neurons assumed to be completely uncorrelated).
\item Probabilistic approaches: Fokker-Planck equation. (Diffusion
  limit).
\item Spatially extended models: description of population of neurons
  in a continuum space. Cortical layers, orientation selectivity,
  whole brain modeling: waves.
\item They provide a framework to be able to compare theoretical
  results with experimental findings. Guide new experimental studies
  based on theoretical findings. 
\item We present a new firing rate model that is exactly derived from
  the dynamics of single neurons, and therefore is able to describe
  novel collective dynamics due to the correlations between neurons,
  such as synchronization, partial synchronization, transient
  synchronization. We show that these collective dynamics 
\end{itemize}


\section{Exact macroscopic description of an all-to-all population of
  QIF neurons (QIF-FR)}

\textit{\color{red} In this section we introduce the QIF-FR model,
  first appeared in \cite{Montbrio2015}. We first give a full
  description of the Quadratic-Integrate-and-Fire (QIF) spiking neuron
model. We then derive the model using the Lorentzian ansatz, as in
\cite{Montbrio2015}, and then present the derivation in the
Theta-neuron space using Ott-Antonsen ansatz.}



\subsection{\sloppy Canonical type-I spiking neuron model:
  Quadratic-Integrate-and-Fire (QIF) model}

\begin{itemize}
\item Description of the full QIF model.
\item Reduction to the normal form of the saddle-node bifurcation.
\item Transfer function of the QIF model (to be used in chapters 2 and 3).
\end{itemize}

% here we describe the whole neuron model with the threshold potential
% and everything else, and we do the reduction to the normal form of
% the saddle-node bifurcation.

\paragraph{Theta neuron:} {\it \color{red} this can be placed in a
  box. Following the style of the text-books.}

\subsection{Dimensionality reduction: Lorentzian Ansatz}


\subsubsection{Ott-Antonsen theory}

\begin{figure}[htbp]
\centerline{\includegraphics[width=\textwidth]{figs/chap1/ott-diagram.pdf}}
\caption[From QIF/Theta to QIF-FR/QIF-Kuramoto]{\label{fig:ott} Scheme
of the derivation of the QIF-Firing Rate model showing different
paths towards the macroscopic description. Starting from the
microscopic description, the dynamics of the population of QIF neurons
may be reduce to a low-dimensional macroscopic description: the QIF-FR
model. Equivalently, the QIF model can be express as an oscillator
model, the Theta neuron, and then derive the low-dimensional
description using the Ott-Antonsen ansatz in the Kuramoto space. From
there the QIF-FR model can be obtained by means of the conformal
mapping, a transformation in the complex plane.}
\end{figure}

\subsection{Firing rate model for current-based spiking neurons}

\begin{itemize}
\item Derivation of the model as in \cite{Montbrio2015}.
\end{itemize}

\subsection{Firing rate model for conductance-based spiking neurons}

Following the general expression  for the QIF neuron
\eqref{eq:qif_general}, we can now define the conductance-based QIF
neuron as:
\begin{equation}
\label{eq:1}
\tau \dot{v}_j = v_j^2 + \eta_j - g \tau S \left( t \right) \left(
  v - v_{E,j} \right),
\end{equation}
where $g$ corresponds to the conductance of the synapses and
$v_{E,j}$ denotes the reversal potential of the neuron $j$, which may
be distributed in a similar fashion as the external currents $\eta_j$.

{\color{red} We explicitly do the derivation ... }
By applying the same procedure as before we obtain the QIF-firing-rate
equations for the population of neurons:
\begin{subequations}
  \label{eq:conductance_fr}
\begin{eqnarray}
\label{eq:conductance_fr-R}
\tau \dot R & = & \frac{\Delta}{\tau \pi} + 2 R V - g \tau R \left( R
                  - \frac{\Gamma }{\tau\pi}\right) \\
  \label{eq:conductance_fr-V}
\tau \dot V & = & V^2 + \bar{\eta} - \left( \pi \tau R \right)^2 - g
                  \tau R \left( V - V_E \right).
\end{eqnarray}
\end{subequations}

\paragraph{Rescaled equations:} rescaling $\Delta$ and $\tau$ in the
following way:
\begin{eqnarray}
\label{eq:3}
R = \frac{\sqrt{\Delta}}{\tau} r; \quad V = \sqrt{\Delta} v; \quad t =
\frac{\tau}{\sqrt{\Delta}}\tilde{t}. \\
\bar{\eta} = \Delta \tilde{\eta}; \quad \Gamma = \Delta \gamma; \quad
V_E = \sqrt{\Delta} v_E, \nonumber
\end{eqnarray}
gives
\begin{subequations}
\label{eq:conductance_fr-rescaled}
\begin{eqnarray}
\label{eq:conductance_fr-R-r}
\dot r & = & \frac{1}{\pi} + 2 rv - g r \left( r
                  - \frac{\gamma}{\pi}\right) \\
  \label{eq:conductance_fr-V-r}
\dot v & = & v^2 + \tilde{\eta} - \left( \pi  r \right)^2 - g
                  r \left( v - v_E \right).
\end{eqnarray}
\end{subequations}




\paragraph{Without external current heterogeneity:}  setting $\Delta =
0$ makes easy to compute the fixed points. We may re-scale the equations
again:
\begin{eqnarray}
\label{eq:3}
R = \frac{\Gamma}{\tau} r; \quad V =  \Gamma v; \quad t =
\frac{\tau}{\Gamma}\tilde{t}. \\
\bar{\eta} = \Gamma^2 \tilde{\eta}; \quad
V_E = \Gamma v_E, \nonumber
\end{eqnarray}
which leads to:
\begin{subequations}
\label{eq:conductance_fr-rescaled2}
\begin{eqnarray}
\label{eq:conductance_fr-R-r-2}
\dot r & = & 2 rv - g r \left( r
                  - \frac{1}{\pi}\right) \\
  \label{eq:conductance_fr-V-r-2}
\dot v & = & v^2 + \tilde{\eta} - \left( \pi  r \right)^2 - g
                  r \left( v - v_E \right).
\end{eqnarray}
\end{subequations}




 Eq. \eqref{eq:conductance_fr-R} gives two possible fixed
points:
\begin{subequations}
  \label{eq:qif-cond-fixed}
  \begin{align}
    \label{eq:qif-cond-fixed1}
    \left(r_{*}, v_{*} \right) &=\left( 0, - \sqrt{-\tilde \eta} \right).\\
    \label{eq:qif-cond-fixed2}
    \left(r_{*}, v_{*} \right) &=\left(\frac{v_Eg \pm \sqrt{\left( v_Eg \right)^2 + \left( g^2 + \pi^2
        \right)\left( 4\eta + g^2/\pi^2 \right)}}{2 +
      \frac{g^2}{2\pi^2}},  \frac{g}{2\pi} \left( \pi r - 1 \right) \right).
  \end{align}
\end{subequations}

\paragraph{Without reversal potential heterogeneity:} 

\section{Neural mass and neural field models}
% ????

%%% Local Variables: 
%%% mode: latex
%%% ispell-local-dictionary: "american"
%%% TeX-master: "main"
%%% End: 
